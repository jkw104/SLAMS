%===========================================================================================
% TEMPLATE FOR STANDARD EXAM
% (with some sample problem types)
% Original by Laura
% Modified by Justin Watson
%===========================================================================================
\documentclass[11pt,epsfig]{article}

\usepackage{amsmath,amssymb}   % Provies math symbols 
\usepackage{mathtools}         % Tools for mathematical typesetting 
\usepackage{graphics}          % Inclusion of graphics in LaTex documents
\usepackage{graphicx}          % Extensions to graphics
\usepackage{scrextend}         % Adds ability to change margin on a block of text.
\usepackage{etoolbox}          % Adds programming facilities 
\usepackage{mhchem}            % Adds ability to correctly write chemical reactions
\usepackage{enumitem}          % Controls layout of itemize, enumerate, etc.

%===========================================================================================
% Create logical variables to turn on or off solutions and graphics elements
%===========================================================================================

\providetoggle{Solution}
\settoggle{Solution}{false}    % false = do not typeset solution; true = typeset solution 

\providetoggle{NoFig}
\settoggle{NoFig}{true}        % true = inlcude graphics; false = do not include graphics

\oddsidemargin=0in
\evensidemargin=0in
\textwidth=6.3in
\topmargin=-.5in
\textheight=9in

\parindent=0in
\pagestyle{empty}

\input{testpoints}

\begin{document}

\iftoggle{Solution}{
  \settoggle{NoFig}{false}     % This is currently set to show certain graphics when 
}                              % solution is true and others when solution is false.

%===========================================================================================
% Cover Sheet
%===========================================================================================

\centerline{\huge \bf Course Title}         
\vfill
\centerline{\huge \bf Exam 1: Topic}        
\vfill \vfill
   
Class Number and Section                     

MM/DD/YYYY \hfill                           % (date of test)
{\bf Name: } $\underbracket[0.5pt][0pt]{\hspace{2.7in}}_{\mbox{\tiny Please print your name clearly}}$
\vfill \vfill \vfill

{\bf Read all of the following information before starting the exam:}
\vspace{1pc}

\begin{itemize}                             % (change info. as desired)
	\item  Show all work, clearly and in order, if you want to get full credit.  I reserve the right to take off points if I cannot see how you 	arrived at your answer (even if your final answer is correct).
	
	\item Justify your answers algebraically whenever possible to ensure 
	full credit. When you do use your calculator (if a calculator is allowed),
	sketch all relevant graphs and explain all relevant mathematics.
	
	\item  Circle or otherwise indicate your final answers.

	\item  Please keep your written answers brief; be clear and to the point.
	I will take points off for rambling and for incorrect or irrelevant 
	statements.
		
	\item  This test has 10 problems        % (number of problems)
	and is worth 100 points,                % (insert total number of points)  
   partial credit will be given for incomplete answers.  If you are
   not sure how to answer a problem give as much information as you can.	
   It is your responsibility to make sure that you have all of the pages!
	
	\item  You will have 1 hour 50 minutes to complete the exam.  
	
	\item  Good luck!
\end{itemize}

\vfill \vfill \vfill

\clearpage

%===========================================================================================
% Question 1
%===========================================================================================

\begin{problem}{8}
\let\clearpage\relax
Label each of the regions of the following flow regime map.  Clearly label the x and y axis including any units and give the range of  values for each of the flow regimes on the x and y axis.  
\iftoggle{NoFig}{
\begin{center}
         \includegraphics[width=3in]{../Figures/FlowMap.png}
\end{center}
}

%
% Solution
%
\iftoggle{Solution}{
  \vspace{12pt}
  \begin{addmargin}[1em]{2em}
    \textbf{Solution:}
    \vspace{12pt}
\begin{center}
         \includegraphics[width=4in]{../Figures/FlowMapSol.png}
\end{center}    
   \end{addmargin}
}


\vfill
\end{problem}

\clearpage

%===========================================================================================
% Question 2
%===========================================================================================

\begin{problem}{8}
\let\clearpage\relax
This problem involves setting up the truncation error analysis for the given fully implicit mass equation with constant positive velocity shown below.

\begin{equation*}
\text{Difference Equation: }\frac{\rho^{n+1}_j-\rho^{n}_j}{\Delta t} + v \frac{\rho^{n+1}_j-\rho^{n+1}_{j-1}}{\Delta x_j} = 0
\end{equation*}

\newpart{2}
 Plot and clearly state the variable that you are making the error evaluation about.  
%
% Solution
%
\iftoggle{Solution}{
  \vspace{12pt}
  \begin{addmargin}[1em]{2em}
    \textbf{Solution:}
    \vspace{12pt}

Use a Taylor series to expand about the point $\rho^{n+1}_j$.  We need one expression for $\rho^{n+1}_{j-1}$ and one for $\rho^n_j$:

\centerline{
  \mbox{
    \includegraphics[width=2in]{../Figures/TaylorSol.pdf}
  }
}
   \end{addmargin}
}
\vfill

\newpart{6}
Write the Taylor series expansion for the two remaining variables about the point chosen above. (NOTE: you don't have to do anything beyond writing the Taylor series expansion.)
%
% Solution
%
\iftoggle{Solution}{
  \vspace{12pt}
  \begin{addmargin}[1em]{2em}
    \textbf{Solution:}
    \vspace{12pt}

\begin{equation*}
   \rho^n_j = \rho^{n+1}_j - \frac{\Delta t}{1!} \left.\frac{\partial \rho}{\partial t}\right|^{n+1}_j + \frac{\Delta t^2}{2!} \left.\frac{\partial^2 \rho}{\partial t^2}\right|^{n+1}_j - \frac{\Delta t^3}{3!} \left.\frac{\partial^3 \rho}{\partial t^3}\right|^{n+1}_j + \cdots
\end{equation*}
\begin{equation*}
   \rho^{n+1}_{j-1} = \rho^{n+1}_j - \frac{\Delta x}{1!} \left.\frac{\partial \rho}{\partial x}\right|^{n+1}_j + \frac{\Delta x^2}{2!} \left.\frac{\partial^2 \rho}{\partial x^2}\right|^{n+1}_j - \frac{\Delta x^3}{3!} \left.\frac{\partial^3 \rho}{\partial x^3}\right|^{n+1}_j + \cdots
\end{equation*}

   \end{addmargin}
}

\vfill

\vfill
\end{problem}

\clearpage

%===========================================================================================
% Scrap Page 
%===========================================================================================
% NOTE: The scrap page and equation sheet are not typeset if solution is true.  
\iftoggle{NoFig}{


{\large \bf Scrap Page}

(please do not remove this page from the test packet)
\clearpage

%===========================================================================================
% Equation Sheet
%===========================================================================================
{\large \bf Useful Equations}
\vspace{12pt}

\begin{equation*}
   f(x) = f(x_0) + \left( x - x_0 \right)\left. \frac{ d f }{ d x } \right|_{x=x_0}  + \left( x - x_0 \right)^2\left. \frac{1}{2} \frac{ d^2 f }{ d x^2 } \right|_{x=x_0}  + \cdots
\end{equation*}

\begin{multline*}
   f(x,y) = f(x_0, y_0) +
   \Delta x \left. \frac{ \partial f }{\partial x } \right|_{x_0, y_0} +
   \Delta y \left. \frac{ \partial f }{\partial y } \right|_{x_0, y_0} +
   \Delta x^2 \frac{1}{2} \left. \frac{ \partial^2 f }{\partial x^2 } \right|_{x_0, y_0} +\\
   \Delta x \Delta y \left. \frac{ \partial^2 f }{ \partial x \partial y } \right|_{x_0, y_0} +
   \Delta y^2 \frac{1}{2} \left. \frac{ \partial^2 f }{\partial y^2 } \right|_{x_0, y_0} + \cdots
\end{multline*}

\begin{multline*}
   g(x,y) = g(x_0, y_0) +
   \Delta x \left. \frac{ \partial g }{\partial x } \right|_{x_0, y_0} +
   \Delta y \left. \frac{ \partial g }{\partial y } \right|_{x_0, y_0} +
   \Delta x^2 \frac{1}{2} \left. \frac{ \partial^2 g }{\partial x^2 } \right|_{x_0, y_0} +\\
   \Delta x \Delta y \left. \frac{ \partial^2 g }{ \partial x \partial y } \right|_{x_0, y_0} +
   \Delta y^2 \frac{1}{2} \left. \frac{ \partial^2 g }{\partial y^2 } \right|_{x_0, y_0} + \cdots
\end{multline*}

\begin{equation*}
\frac{\partial \rho}{\partial t}+\nabla\cdot(\rho \vec{v})=0
\end{equation*}

\begin{equation*}
\frac{\partial \rho e}{\partial t}+\nabla\cdot(\rho e\vec{v})+P\nabla\cdot\vec{v}=q_{wl}
\end{equation*}

\begin{equation*}
\frac{\partial \vec{v}}{\partial t}+\vec{v}\nabla\cdot(\vec{v})+\frac{1}{\rho}\nabla P + K\vec{v}^2=0
\end{equation*}

\begin{equation*}
         r = \frac{ h_3 }{ h_2 } = \frac{ h_2 }{ h_1 } 
\end{equation*}

\begin{align*}
   &f_1 = f_{exact} + a h^p_1\\
   &f_2 = f_{exact} + a h^p_2\\
   &f_3 = f_{exact} + a h^p_3
\end{align*}

\begin{equation*}
   \frac{ f_3 - f_2 }{ f_2 - f_1 } = \frac{ h^p_2 }{ h^p_1 } = r^p
\end{equation*}

\begin{equation*}
   p = \frac{ \ln \left( \frac{ f_3 - f_2 }{ f_2 - f_1 } \right) }{ \ln \left( r \right) }
\end{equation*}

\begin{equation*}
   f_{exact} = f_1 - \frac{ f_2 - f_1 }{ r^p - 1 }
\end{equation*}

\begin{equation*}
\frac{1}{2} \rho v^2_1 + P_1 = \frac{1}{2} \rho v^2_2 + P_2 + \Delta P_l
\end{equation*}

\begin{equation*}
\Delta P_l = 
\frac{1}{2} \rho v^2_1 
\left(
1
-\frac{A_1}{A_2}
\right)^2
\end{equation*} 


}

\showpoints
\end{document}


